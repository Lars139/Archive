
%% Support sites:
%% http://www.michaelshell.org/tex/ieeetran/
%% http://www.ctan.org/pkg/ieeetran
%% and
%% http://www.ieee.org/

%%*************************************************************************
%% Legal Notice:
%% This code is offered as-is without any warranty either expressed or
%% implied; without even the implied warranty of MERCHANTABILITY or
%% FITNESS FOR A PARTICULAR PURPOSE! 
%% User assumes all risk.
%% In no event shall the IEEE or any contributor to this code be liable for
%% any damages or losses, including, but not limited to, incidental,
%% consequential, or any other damages, resulting from the use or misuse
%% of any information contained here.
%%
%% All comments are the opinions of their respective authors and are not
%% necessarily endorsed by the IEEE.
%%
%% This work is distributed under the LaTeX Project Public License (LPPL)
%% ( http://www.latex-project.org/ ) version 1.3, and may be freely used,
%% distributed and modified. A copy of the LPPL, version 1.3, is included
%% in the base LaTeX documentation of all distributions of LaTeX released
%% 2003/12/01 or later.
%% Retain all contribution notices and credits.
%% ** Modified files should be clearly indicated as such, including  **
%% ** renaming them and changing author support contact information. **
%%*************************************************************************


% *** Authors should verify (and, if needed, correct) their LaTeX system  ***
% *** with the testflow diagnostic prior to trusting their LaTeX platform ***
% *** with production work. The IEEE's font choices and paper sizes can   ***
% *** trigger bugs that do not appear when using other class files.       ***                          ***
% The testflow support page is at:
% http://www.michaelshell.org/tex/testflow/



\documentclass[journal]{IEEEtran}


% *** GRAPHICS RELATED PACKAGES ***
%
\ifCLASSINFOpdf
  % \usepackage[pdftex]{graphicx}
  % declare the path(s) where your graphic files are
  % \graphicspath{{../pdf/}{../jpeg/}}
  % and their extensions so you won't have to specify these with
  % every instance of \includegraphics
  % \DeclareGraphicsExtensions{.pdf,.jpeg,.png}
\else
  % or other class option (dvipsone, dvipdf, if not using dvips). graphicx
  % will default to the driver specified in the system graphics.cfg if no
  % driver is specified.
  % \usepackage[dvips]{graphicx}
  % declare the path(s) where your graphic files are
  % \graphicspath{{../eps/}}
  % and their extensions so you won't have to specify these with
  % every instance of \includegraphics
  % \DeclareGraphicsExtensions{.eps}
\fi
% graphicx was written by David Carlisle and Sebastian Rahtz. It is
% required if you want graphics, photos, etc. graphicx.sty is already
% installed on most LaTeX systems. The latest version and documentation
% can be obtained at: 
% http://www.ctan.org/pkg/graphicx
% Another good source of documentation is "Using Imported Graphics in
% LaTeX2e" by Keith Reckdahl which can be found at:
% http://www.ctan.org/pkg/epslatex
%
% latex, and pdflatex in dvi mode, support graphics in encapsulated
% postscript (.eps) format. pdflatex in pdf mode supports graphics
% in .pdf, .jpeg, .png and .mps (metapost) formats. Users should ensure
% that all non-photo figures use a vector format (.eps, .pdf, .mps) and
% not a bitmapped formats (.jpeg, .png). The IEEE frowns on bitmapped formats
% which can result in "jaggedy"/blurry rendering of lines and letters as
% well as large increases in file sizes.
%
% You can find documentation about the pdfTeX application at:
% http://www.tug.org/applications/pdftex





% *** MATH PACKAGES ***
%
%\usepackage{amsmath}
% A popular package from the American Mathematical Society that provides
% many useful and powerful commands for dealing with mathematics.
%
% Note that the amsmath package sets \interdisplaylinepenalty to 10000
% thus preventing page breaks from occurring within multiline equations. Use:
%\interdisplaylinepenalty=2500
% after loading amsmath to restore such page breaks as IEEEtran.cls normally
% does. amsmath.sty is already installed on most LaTeX systems. The latest
% version and documentation can be obtained at:
% http://www.ctan.org/pkg/amsmath





% *** SPECIALIZED LIST PACKAGES ***
%
%\usepackage{algorithmic}
% algorithmic.sty was written by Peter Williams and Rogerio Brito.
% This package provides an algorithmic environment fo describing algorithms.
% You can use the algorithmic environment in-text or within a figure
% environment to provide for a floating algorithm. Do NOT use the algorithm
% floating environment provided by algorithm.sty (by the same authors) or
% algorithm2e.sty (by Christophe Fiorio) as the IEEE does not use dedicated
% algorithm float types and packages that provide these will not provide
% correct IEEE style captions. The latest version and documentation of
% algorithmic.sty can be obtained at:
% http://www.ctan.org/pkg/algorithms
% Also of interest may be the (relatively newer and more customizable)
% algorithmicx.sty package by Szasz Janos:
% http://www.ctan.org/pkg/algorithmicx




%\usepackage{array}
% Frank Mittelbach's and David Carlisle's array.sty patches and improves
% the standard LaTeX2e array and tabular environments to provide better
% appearance and additional user controls. As the default LaTeX2e table
% generation code is lacking to the point of almost being broken with
% respect to the quality of the end results, all users are strongly
% advised to use an enhanced (at the very least that provided by array.sty)
% set of table tools. array.sty is already installed on most systems. The
% latest version and documentation can be obtained at:
% http://www.ctan.org/pkg/array

% IEEEtran contains the IEEEeqnarray family of commands that can be used to
% generate multiline equations as well as matrices, tables, etc., of high
% quality.




% *** SUBFIGURE PACKAGES ***
%\ifCLASSOPTIONcompsoc
%  \usepackage[caption=false,font=normalsize,labelfont=sf,textfont=sf]{subfig}
%\else
%  \usepackage[caption=false,font=footnotesize]{subfig}
%\fi
% subfig.sty, written by Steven Douglas Cochran, is the modern replacement
% for subfigure.sty, the latter of which is no longer maintained and is
% incompatible with some LaTeX packages including fixltx2e. However,
% subfig.sty requires and automatically loads Axel Sommerfeldt's caption.sty
% which will override IEEEtran.cls' handling of captions and this will result
% in non-IEEE style figure/table captions. To prevent this problem, be sure
% and invoke subfig.sty's "caption=false" package option (available since
% subfig.sty version 1.3, 2005/06/28) as this is will preserve IEEEtran.cls
% handling of captions.
% Note that the Computer Society format requires a larger sans serif font
% than the serif footnote size font used in traditional IEEE formatting
% and thus the need to invoke different subfig.sty package options depending
% on whether compsoc mode has been enabled.
%
% The latest version and documentation of subfig.sty can be obtained at:
% http://www.ctan.org/pkg/subfig




% *** FLOAT PACKAGES ***
%
%\usepackage{fixltx2e}
% fixltx2e, the successor to the earlier fix2col.sty, was written by
% Frank Mittelbach and David Carlisle. This package corrects a few problems
% in the LaTeX2e kernel, the most notable of which is that in current
% LaTeX2e releases, the ordering of single and double column floats is not
% guaranteed to be preserved. Thus, an unpatched LaTeX2e can allow a
% single column figure to be placed prior to an earlier double column
% figure.
% Be aware that LaTeX2e kernels dated 2015 and later have fixltx2e.sty's
% corrections already built into the system in which case a warning will
% be issued if an attempt is made to load fixltx2e.sty as it is no longer
% needed.
% The latest version and documentation can be found at:
% http://www.ctan.org/pkg/fixltx2e


%\usepackage{stfloats}
% stfloats.sty was written by Sigitas Tolusis. This package gives LaTeX2e
% the ability to do double column floats at the bottom of the page as well
% as the top. (e.g., "\begin{figure*}[!b]" is not normally possible in
% LaTeX2e). It also provides a command:
%\fnbelowfloat
% to enable the placement of footnotes below bottom floats (the standard
% LaTeX2e kernel puts them above bottom floats). This is an invasive package
% which rewrites many portions of the LaTeX2e float routines. It may not work
% with other packages that modify the LaTeX2e float routines. The latest
% version and documentation can be obtained at:
% http://www.ctan.org/pkg/stfloats
% Do not use the stfloats baselinefloat ability as the IEEE does not allow
% \baselineskip to stretch. Authors submitting work to the IEEE should note
% that the IEEE rarely uses double column equations and that authors should try
% to avoid such use. Do not be tempted to use the cuted.sty or midfloat.sty
% packages (also by Sigitas Tolusis) as the IEEE does not format its papers in
% such ways.
% Do not attempt to use stfloats with fixltx2e as they are incompatible.
% Instead, use Morten Hogholm'a dblfloatfix which combines the features
% of both fixltx2e and stfloats:
%
% \usepackage{dblfloatfix}
% The latest version can be found at:
% http://www.ctan.org/pkg/dblfloatfix


%\ifCLASSOPTIONcaptionsoff
%  \usepackage[nomarkers]{endfloat}
% \let\MYoriglatexcaption\caption
% \renewcommand{\caption}[2][\relax]{\MYoriglatexcaption[#2]{#2}}
%\fi
% endfloat.sty was written by James Darrell McCauley, Jeff Goldberg and 
% Axel Sommerfeldt. This package may be useful when used in conjunction with 
% IEEEtran.cls'  captionsoff option. Some IEEE journals/societies require that
% submissions have lists of figures/tables at the end of the paper and that
% figures/tables without any captions are placed on a page by themselves at
% the end of the document. If needed, the draftcls IEEEtran class option or
% \CLASSINPUTbaselinestretch interface can be used to increase the line
% spacing as well. Be sure and use the nomarkers option of endfloat to
% prevent endfloat from "marking" where the figures would have been placed
% in the text. The two hack lines of code above are a slight modification of
% that suggested by in the endfloat docs (section 8.4.1) to ensure that
% the full captions always appear in the list of figures/tables - even if
% the user used the short optional argument of \caption[]{}.
% IEEE papers do not typically make use of \caption[]'s optional argument,
% so this should not be an issue. A similar trick can be used to disable
% captions of packages such as subfig.sty that lack options to turn off
% the subcaptions:
% For subfig.sty:
% \let\MYorigsubfloat\subfloat
% \renewcommand{\subfloat}[2][\relax]{\MYorigsubfloat[]{#2}}
% However, the above trick will not work if both optional arguments of
% the \subfloat command are used. Furthermore, there needs to be a
% description of each subfigure *somewhere* and endfloat does not add
% subfigure captions to its list of figures. Thus, the best approach is to
% avoid the use of subfigure captions (many IEEE journals avoid them anyway)
% and instead reference/explain all the subfigures within the main caption.
% The latest version of endfloat.sty and its documentation can obtained at:
% http://www.ctan.org/pkg/endfloat
%
% The IEEEtran \ifCLASSOPTIONcaptionsoff conditional can also be used
% later in the document, say, to conditionally put the References on a 
% page by themselves.




% *** PDF, URL AND HYPERLINK PACKAGES ***
%
%\usepackage{url}
% url.sty was written by Donald Arseneau. It provides better support for
% handling and breaking URLs. url.sty is already installed on most LaTeX
% systems. The latest version and documentation can be obtained at:
% http://www.ctan.org/pkg/url
% Basically, \url{my_url_here}.


% correct bad hyphenation here
\hyphenation{op-tical net-works semi-conduc-tor}


\begin{document}
\title{ECE472: Computer Architecture\\ Homework 4: Part1}
\author{Rattanai Sawaspanich}

% make the title area
\maketitle

\section{Memory Optimization (PowerPoint)}
 
% Here we have the typical use of a "T" for an initial drop letter
% and "HIS" in caps to complete the first word.
\IEEEPARstart{I}{n} the past twenty years, the speed of a CPU has increased 
apporximately 60\% per year while the memory speed has only increased by 
about 10\% a year. To close the performance gap, the memory read/write needs
to be optimized from a user persepctive.\\

% needed in second column of first page if using \IEEEpubid
%\IEEEpubidadjcol

\subsection{Terminology}
\textbf{Caches} is a local memory that can be access at a faster speed than
a generic memory. There are two type of caches in general: Instruction cache
and Data cache. \textbf{Instruction cache} is for instruction set and opcodes
to be stored. \textbf{Data cahce} is for program data to be store. 
\newline
\newline
\textbf{Cache Lines} are lines of memory where caches are divided into either32 or 64 byte sections.
\textbf{Direct-Mapping} is a method to map a memory to a corresponding
cache line.
\textbf{N-way set-associative} is a way to map a logical cache line to the
physical cache line (a bus).

\subsection{Architecture Overview}
Caches are typically divided into different levels e.g. L1, L2, and 
main memory.
\newline
\newline
\textbf{L1 cache} is the fastest not-on-CPU memory that a CPU can access. 
Generally L1 cache has a fairly small storage size in the magnitude of ~8KB
and takes about 1 clock cycle to access a memory in L1 cache. Note: the 
smaller size storage is due to the physical hardware limitation.
\newline
\newline
\textbf{L2 cache} is slower cache than L2. Though, it has a larger storage
that can go up the magnitue of 8MB. On the average, it takes a CPU about
5 to 20 clock cycles to access a memory in an L2 cache. 
Let's assume the system is a 2-level cache heirachy. After all the caches 
have been depleted, the CPU needs to access the main memory.
\newline
\newline
\textbf{Main memory} or \textbf{system memory} is the biggest and slow source
of memory. The storage magnitude of the system memory can be in the order of
gigabyt to terabyte. It usually takes a CPU about 40 to 100 clock cycles to
access the memory. 
\newline
\newline
\subsection{Optimization}
\subsubsection{Causes of cache misses}
\begin{description}
   \item\textbf{Compulsory Misses} are invitable cache misses occur when the
      data is read for the first time.
   \item\textbf{Capacity Misses} are cache misses occur when the active data
      requires more memory than a cache storage. This type of miss usually
      happens when there is too much data accessed.
   \item\textbf{Conflict Misses} are cache misses due to the cache line 
      conflict -- a cache mapped to the same cache line. Note: a cache line 
      can only handle a set of data at a time.
      \newline
\end{description}

\subsubsection{Key to Optimization}
\begin{description}
   \item\textbf{Reuse} the set of data (or variables) to 
      increase the temporal and sparial locality which helps avoiding 
      multiple compulsory cache misses.
   \item\textbf{Rearrange} the order how the data is being stored to change
      the layout and increase the spatial locality. The rearrangement helps
      with the capacity cache misses.
   \item\textbf{Reduce} the number of a cache line read to help with cache
      conflict misses -- the lesser accessed data, the lesser cache collision
      there will be.
      \newline
\end{description}

\subsubsection{Optimization Techniques}
\textbf{\\Reuse: Prefetch}
\begin{description}
   \item\textbf{Software Prefetching} -- use an arbitrary look-ahead to 
      prefecth the data in cache before needed by CPU
   \item\textbf{Greedy Prefetching} -- prefetch all the data that will be 
      used in the next two instructions
   \item\textbf{Preloading} -- ask for a data to be used and while it is 
      waiting for the memory I/O has a CPU perform  other tasks first.
\end{description}

\textbf{Rearrange: Structures}
\begin{description}
   \item\textbf{Cache-conscious layout} -- rearrange the data that typically
      access together next to each other i.e. declare the variables next to
      each other, or using array declaration. 
   \item\textbf{Use structures} -- use tree structure to increase spatial 
      locality and reduce the size of pointer used by storing data relative
      to the parent node e.g. Breadth-first order, Depth-frist order, Van 
      Emde Boas layout, compact static k-d tree.
      \newline
      \newline
   \item\textbf{Linearlized data} -- store and fetch data in a linear manner
      using static memory pool. The method yields the best performance. The
      linearization technique can be used on top of hierarchy data 
      structures to prevent blocking overhead and keep data together with
      minimal fragmentation. It is highly recommended to allocate memory 
      pool from stack, free the block immediate if not needed, and reuse the
      memory if possible. 
\end{description}

\subsection{Alising}
\textbf{Alising} occurs when multiple references point to the same memory 
location. This can prevent compiler from optimizing the code by prohibiting
the reordering (badly impact instruction scheduling), eliminating of loads 
(hinder looping optimization and subexpression elimination), 
and storing of the data.

\subsubsection{Causes of Alising}
\begin{description}
   \item\textbf{Pointers} are the major causes of alising specially shared 
      pointers.  A compiler does not know how to optimize the program 
      because the value stores in the pointed location can be changed 
      any time at runtime. A compiler does not have access to the runtime
      values; there is no optimization at compile time.
      \newline

   \item\textbf{Global variables/ Class Members} can cause aliasing because
      at the end they can be resolved as shared pointers.\newline
\end{description}

\subsubsection{Aliasing Prevention}
\begin{description}
   \item\textbf{Program Better:} When a CPU fetch information from a cache,
      the whole cache line is returned. Anti-aliasing can be done using the
      fact, the data only needs to be fetched once for every values in the
      same row (assuming a row major language).
   \item\textbf{Lower Language Abstraction:} The more abstract, the more 
      abstraction penalty there is. The code can only breakdown to at most
      a generic code block which prevents any special customization. 
      Moreover, an abstraction takes away the transparency of data and its 
      operations e.g. temporary objects, implicit aliasing pointers, etc. 
      This can introduce aliasing issues.
   \item\textbf{Use a Better Compiler} some compiler is better at 
      anti-aliasing than the other. Note: use –fstrict-aliasing in GCC
      to turn on anti-aliasing feature. 
      \newline
   \item\textbf{Potpourri}
      \begin{itemize}
	 \item Minimize use of pointers and global variables.
	 \item Use local variables as much as possible.
	 \item Avoid operation that requires temporary variables.
	 \item Declare variables close to where they are used.
	 \item Use 'const'
      \end{itemize}
\end{description}
%------------------------------------------------------- PAPER

\section{What Every Programmer Should Know About Memory}
\subsection{The Current Hardware}
\textbf{Personal Computer System\\}
In the past twenty years, personal computers have become more popular. 
The chipset of such computer has divided into two parts: the Northbride and 
the Southbride.  All the CPU are connected via a common bus to the 
Northbridge. \textbf{The Northbride} takes part in memory control all 
different type of RAM. \textbf{The Southbride} acts as an I/O bridge 
handling the communication with different devices via peripheral buses e.g.
PCI, PCIe, SATA, and USB buses.\newline

It is worth note taking that all of the data communcation from CPU to CPU
,and all RAM communication can only be done through Northbridge. This can 
cause a bottleneck in the I/O data stream. This is not to mention that RAM
in the system is a single full-duplex port. Instantly, there are two 
bottleneck spots in the system. The first bottleneck occurs when RAM tries
to communicate with the different devices via the bridges. The data has to 
route through the CPU which can slow down the entire sytem. The second
bottleneck occurs when a bus from the Northport tries to communicate with 
RAM. \newline

\textbf{Advanced Computer System\\}
On the more expensive and advanced computer system, the Northbride does not
contain only a single memory control; rather, a number of external memory
controllers. The more memory controllers allow a higher bandwidth to be 
used in the system. Though, multiple external memory controller is not the 
only solution to increase the memory I/O bandwidth. Another way to increase
the system bandwidth is by integrating the memory controller in the CPU 
itself along with an integrated on-chip memory. \newline

There is no free lunch in a more advanced system. The multi-memory 
controllers system needs make all of its memory accessible to all the 
processors. Moreover, not all of the memory can be accessed with the same
constant time. The scheme name for this multi-memory controller is called
\textbf{NUMA} or \textbf{Non-Uniform Memory Architecture}. The reason behind
its non-uniform accessing time is because the on-chip local memory can be
accessed with the usual speed while the memory that is attached to a 
different processor needs to be transfer via the interconnect buses.

\subsection{Types of RAM}
There are basically two types of RAM: Static RAM (SRAM) and Dynamic RAM 
(DRAM).  SRAM is a lot faster at accessing the memory than DRAM. The only
reason why DRAM exists is because of the cost to produce SRAM.\newline 

\textbf{Static RAM (SRAM)\\}
A unit of SRAM is built off of six transistors that forms two cross-coupled
inverters. Though, in some designs, only four transistors are needed.
These cell state is ready to be read at almost instant given the cell state
does not need a refresh cycle. There are some cons to this type of RAM is 
that it requires a constant supply of power to maintain its state.\newline

\textbf{Dynamic RAM (DRAM)\\}
A unit of DRAM is comprised of one transistor and one capacitor. The 
transistor is used to guard the access state. The reading process of DRAM
is to discharge the memory cell. This task cannot be repeated indefinitely,
the capacitor wears out and can have electrons leakage. Moreover DRAM 
cannot be instantly read/write as if SRAM, it requires sensing amplifier to 
translate the voltage into either 1 or 0.\newline

%\subsubsection{Type of DRAM}

\subsection{CPU Caches}
To obtain a high performance system, the memory I/O should hypothetically 
run at the same speed as CPU to yield the maximum performance. Though, it is
not possible due to the law of physics -- a capacitor in DRAM takes 
time to charge up.  Fortunately, SRAM can operate at a much faster speed than
DRAM. In this case, SRAM can be use as a medium ground for a DRAM to 
communicate with the CPU.  Though, it is not a viable option to have a 
section of memory as SRAM and the rest DRAM due to the complication in 
logical address mapping. So instead of hanving a giant pool of mixed 
SRAM-DRAM memory under the control of an operating system or a user, SRAM
can be used an internal memory for a processor to use. It is used to 
make a temporary copies of data in main meory which is likely to be used
soon by the processor.  The on-processor SRAM is called a cache. \newline

Typically the size of SRAM caches is many times smaller than the system
memory to keep the cost down and prevent crowded effects. In the cache 
configuration module, CPU is no longer directly connected to main memory. 
All the data needs to be loaded and go through caches first before entering
the CPU. More recently, an instruction cache is introduced to maximize the 
instruction decoding processes. At this point there is a huge speed gap 
between an internal processor SRAM and system memory DRAM. To close the speed
gap, more level of caches were introduces. The higher the bigger storage size
, but the slower it is.\newline

In the more recent design, each processor can have multiple cores and each
core can have multiple threads. Note: the only different between thread and 
core is that a separate cores
have separate copies of all the hardware resources and can run completely
independent from each other. On the other hand, threads shared almost all of
the processor's resources. The cache storage needs to be able to 
facilitate the needs of memory resources for all the threads and cores. 

\subsection{Caches Operation}
If the CPU needs a data word that cahes are searched first, the cache 
cannot have the content of the entire memory, but since all the memory is 
cacheable, each cahe entry is tagged using the address of the data word 
in the main memory. Given the concept of spatial locality is one of the 
principles on cahces, the neighbor memory is likely to be used together; 
it should also be cached. When the data in cache is need by the processor,
the entire cache line is loaded into the L1 cache which forwards the whole 
thing to the processor. \newline

In some cases, there is a chance that some information needed by the 
processor is not stored in a cache (cache misses). The processor needs to
fetch the entire new data from system memory. 

\subsection{Associativity Cache}
It is possible to have a cache where each cache line holds a copy of any 
memory location (also known as \textbf{fully associative cache}) by tagging
the entire part of the address which is not offset into the cache line. 
The only cache to this method is the storage needs to store the tags lookup
would be gigantic given the current standard size of L1 and L2.
\newline

\subsection{Measurement of Cache Effects}
The simplest case is a simple walk over all the entires in the list. The list
element are laid out sequentially. What needs to be measured here is how long
does it take to access a single list element. The number of workload where
there is a spike in the element accessed time is the point where the workload
exceeds that cache level capacity, so the processor needs to fetch the 
information from the next cache level which causes the increases in accessing
time.








\end{document}


