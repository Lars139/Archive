\documentclass[letterpaper,10pt,titlepage,fleqn]{article}

%example of setting the fleqn parameter to the article class -- the below sets the offset from flush left (fl)
\setlength{\mathindent}{1cm}

\usepackage{graphicx}

\usepackage{amssymb}
\usepackage{amsmath}
\usepackage{amsthm}

\usepackage{alltt}
\usepackage{float}
\usepackage{color}

\usepackage{url}

\usepackage{balance}
\usepackage[TABBOTCAP, tight]{subfigure}
\usepackage{enumitem}

\usepackage{pstricks, pst-node}

%the following sets the geometry of the page
\usepackage{geometry}
\geometry{textheight=9in, textwidth=6.5in}

% random comment

\newcommand{\cred}[1]{{\color{red}#1}}
\newcommand{\cblue}[1]{{\color{blue}#1}}

\usepackage{hyperref}

\usepackage{textcomp}
\usepackage{listings}

\def\name{Rattanai Sawaspanich}

%% The following metadata will show up in the PDF properties

\hypersetup{
  colorlinks = true,
  urlcolor = black,
  pdfauthor = {\name},
  pdfkeywords = {ECE472 Advance Computer Architecture},
  pdftitle = {ECE472 Homework 1},
  pdfsubject = {ECE472 Homework 1},
  pdfpagemode = UseNone
}

\parindent = 0.0 in
\parskip = 0.2 in

\begin{document}

\pagestyle{empty}

\begin{center}
\section*{ECE472 Advance Computer Architecture - Homework 1}
Write-up

Rattanai Sawaspanich
\end{center}

%\rule[0 in]{6.5 in}{0.01 in}

\section*{Describe the difference between architecture and organization.}

A computer \textbf{architecture} is typically put more empahsis on software aspect of a computer; it is defined by the assembly instruction set that a CPU can support. 
While an \textbf{organization} focuses more on hardware aspect and is defined by how transistors, registers, memory, are laid out physically.

\section*{Describe the concept of endianness. What common platforms use what endianness?}

The concept of endianness is used to describe how each byte in a word is enumerated. 
\textbf{Big endian} is a bit-oriented byte ordering that puts the most significant byte (the highest byte value) in the right-most byte position of a word and the lowest byte value in the left-most byte position. The Big Endian is a byte oriented because the bytes in a word is numbered in order. 
\textbf{Little endian} is a byte-oriented byte ordering that puts the least significant byte (the lowest byte value) in the right-most bit position of a word and the highest byte value in the left-most byte position. The Little Endian enriches the bit-orientation because all the bits in a word is in a decending order.
\newline 32 bits architecture: \newline
\begin{tabular} {l c  c  c  c}
   Big   Endian:   & 07|00    &  15|08    & 23|16    & 31|24\\ 
   Bit             & [7 ... 0]&  [7 ... 0]& [7 ... 0]& [7 ... 0]\\ 
   Small Endian:   & 31|24    &  23|16    & 15|08    &  07|00 \\
\end{tabular}
\newline
Note: x86 and ARM architecture use little endian format.

\section*{Give the IEEE 754 floating point format for both single and double precision.}

\begin{center}
   $ Float Number = (-1)^{SignBit} * (1 + Fraction) * 2^{Exponent - Bias}  $

\begin{tabular} { l | c | c }
   Variable & Single  & Double\\
   Fraction & 23 bits & 52 bits\\ 
   Exponent & 8 bits  & 11 bits\\
   Bias     & 127 bits& 1023 bits \\

\end{tabular}
\end{center}

\section*{Describe the concept of the memory hierarchy. What levels of the hierarchy are present on flip.engr.oregonstate.edu?}

Memory heirarchy are layers of computer storage/memory. Each layer represents a blocks of storage that has the same accessing time that is unique to the layer (or level). \newline

Flip Memory Heirarchy:\newline
1st-level data cache  24 KBytes, 6-way set associative, 64 byte line size\newline
 No 2nd-level cache or, if processor contains a valid 2nd-level cache, no 3rd-level cache
 

\section*{What streaming SIMD instruction levels are present on flip.engr.oregonstate.edu?}

SSE3, SSSE3, SSE4.1, SSE, SSE2 are streaming SIMD instruction leveles that are presented on flip.engr.oregonstate.edu  

\section*{If a feature is not supported, report that.}

Since the maximum number of CPUID leaf that is supported on flip.engr.oregonstate.edu is 10, any features that has a higher leaf than 10 is not supported. For this specific assignment, the clock frequency feature is not supported. [Thanks to CPU Brand name string that provide us with CPU frequency].


\end{document}

