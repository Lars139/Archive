\documentclass{article}
\usepackage{ShumanNotes}
\newcommand{\tab}{\hspace*{3em}}
\title{Homework 2: Materials and Devices}
\author{Rattanai Sawaspanich}
\date{Due October 11th at 3 PM, 2012}
\setlength{\parindent}{0pt}
\pdfpagewidth 8.5in
\pdfpageheight 11in

\begin{document}
\maketitle

\section{Resistor Color Codes}
\emph{A resistor has the color stripes of red black red gold. What range of resistances could this represent?}
\newline
Modify this line to represent the correct equation for changing the color code to resistance:\newline
Equation:  $color_{1}*10^{color_{3}+1}+color_{2}*10^{color_{3}} \pm color_{4}= resistance \pm error$\newline
Modified Equation:  $2*10^{3}+0*10^{2} \pm 5\% = 2*10^{3} \pm 5\%$  ohm \newline
Therefore, the range of the resistor will be $1.9*10^3$  to $ 2.1*10^3$ ohm

\section{OSU Solar Vehicle Team}
\emph{Research the OSU Solar Vehicle Team.}  \newline
Group Website:  \url{http://people.oregonstate.edu/groups/solar/}  \newline
Video:  \url{http://www.youtube.com/watch?v=8jwVyBbDfjY}\newline\newline
\tab Before the Pheonix project began, the OSUSVT has to deal with something explosive; the battery of the solar car explode while driving on the 15th Street and  the solar car was gone.  The Phoenix project began. To make a better solar car, the OSUSVT focuses on three keywords: safer, quicker, and more efficient.  Safter-- the lithium nano-iron phosphate cells are used as the battery pack. Quicker -- Using the Egress system. Efficient - improve break, wheel cover, and luminator. Moreover, the solar car body was made by Carbon-fiber which is durable and light weight that is a factor to help improve the efficiency.  In general Phoenix solar car includes with 1,032 solar panels which makes of crystal silicon and can generate approximately one kilowatts on full sun\footnote{The intensity of the full sun has range between 32,000 to 130,000 lux}.  At the rate, the solar car can move with the speed of 40 mph.  For the controlling system, Phoenix uses its unique program that was created from Visual Basic Developer and display on a Windows7-tablet as a surface of the steerling wheel.

\begin{enumerate}
\item How efficient are these solar panels?  why are they flatening? I have heard shape does matter on how efficient the solar panels will be.
\item What would be the most practical way of getting a driver in a solar car without going under and then sit in the seat?  

\end{enumerate}

\cfoot{ECE 111 Homework \#2}

\end{document}
