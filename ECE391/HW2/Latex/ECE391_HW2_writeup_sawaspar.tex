\documentclass[letterpaper,10pt,titlepage,fleqn]{article}

%example of setting the fleqn parameter to the article class -- the below sets the offset from flush left (fl)
\setlength{\mathindent}{1cm}

\usepackage{graphicx}

\usepackage{amssymb}
\usepackage{amsmath}
\usepackage{amsthm}

\usepackage{alltt}
\usepackage{float}
\usepackage{color}

\usepackage{url}

\usepackage{balance}
\usepackage[TABBOTCAP, tight]{subfigure}
\usepackage{enumitem}

\usepackage{pstricks, pst-node}

%the following sets the geometry of the page
\usepackage{geometry}
\geometry{textheight=9in, textwidth=6.5in}

% random comment

\newcommand{\cred}[1]{{\color{red}#1}}
\newcommand{\cblue}[1]{{\color{blue}#1}}

\usepackage{hyperref}

\usepackage{textcomp}
\usepackage{listings}

\def\name{Rattanai Sawaspanich}

%% The following metadata will show up in the PDF properties

\hypersetup{
  colorlinks = true,
  urlcolor = black,
  pdfauthor = {\name},
  pdfkeywords = {ECE391: Transmission Line},
  pdftitle = {Special Homework 2},
  pdfsubject = {CS 444 Project 2},
  pdfpagemode = UseNone
}

\parindent = 0.0 in
\parskip = 0.2 in

\begin{document}

\pagestyle{empty}

\begin{center}
\section*{ECE391: Transmission Line}
Special Homework 2

Rattanai Sawaspanich
\end{center}

\section*{Initial Thoughts}
I honestly did not know what was going on with the curcuit so I started by 
trying to filer our different range of frequency from the circuit. I traced
it down to find out that the \textit{endpnt2} was actually the one causing 
a lot of resonating signals.  Once I knew what was causing the issue, I started
to instantiate different method to solve the problem. 

\section*{Design 1: Using a LC-resonator}
I noticed that the noises generated by \textit{endpnt2} looks really similar
to a signal that would be genearted by having an inductor and a capacitor in
parallel. So,the first design that came to my mind was using an LC-resonator 
to cancle the noises out. A 15pF-capacitor in series with a 25pH-inductor was
added in parallel with the load at \textit{endpt2}. To handle an echo from the
resonator itself, another resonator is added to \textit{endpnt1} was added
to handle the echo. It was a 25pH-inductor in series with 5pF-capacitor 
connected to ground.

\section*{Design 2: Using capacitors to SHORT everything}
This method used holes in the rules homework assignment to basically attach
a capacitor in parallel with T-line just to short the T-line out. This is
basically getting rid of the T-line from the entire circuit. It is 
technically legal, but practically infeasible. 

\section*{Design 3: Using passive component and a capacitor}
This is the finalized module that encompasses both passive and reactive 
component. This method turned out the be the best in term of perform. It
can provide a decently fast rising time with a few resonation noise. 
Resistors were added in series at the beginning of \textit{endpnt1} and 
\textit{endpnt2} to match the impedance of the endpoints with the T-lines
which prevents any signal resonant to happen. Though the two resistors 
provide a very clear waveform, there are some overshoot and ringing of 
at the edege of T-line. I notice that most of the spiking of the T-line
was causing by lower frequency signals. To handle the issue, a capacitor 
was added to behave as a high-pass filter, getting rid of any lower frequency 
signals that cuases the ringing at the end of T-line. 




\end{document}

